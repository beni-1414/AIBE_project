% Options for packages loaded elsewhere
\PassOptionsToPackage{unicode}{hyperref}
\PassOptionsToPackage{hyphens}{url}
%
\documentclass[
]{article}
\usepackage{amsmath,amssymb}
\usepackage{iftex}
\ifPDFTeX
  \usepackage[T1]{fontenc}
  \usepackage[utf8]{inputenc}
  \usepackage{textcomp} % provide euro and other symbols
\else % if luatex or xetex
  \usepackage{unicode-math} % this also loads fontspec
  \defaultfontfeatures{Scale=MatchLowercase}
  \defaultfontfeatures[\rmfamily]{Ligatures=TeX,Scale=1}
\fi
\usepackage{lmodern}
\ifPDFTeX\else
  % xetex/luatex font selection
\fi
% Use upquote if available, for straight quotes in verbatim environments
\IfFileExists{upquote.sty}{\usepackage{upquote}}{}
\IfFileExists{microtype.sty}{% use microtype if available
  \usepackage[]{microtype}
  \UseMicrotypeSet[protrusion]{basicmath} % disable protrusion for tt fonts
}{}
\makeatletter
\@ifundefined{KOMAClassName}{% if non-KOMA class
  \IfFileExists{parskip.sty}{%
    \usepackage{parskip}
  }{% else
    \setlength{\parindent}{0pt}
    \setlength{\parskip}{6pt plus 2pt minus 1pt}}
}{% if KOMA class
  \KOMAoptions{parskip=half}}
\makeatother
\usepackage{xcolor}
\usepackage[margin=1in]{geometry}
\usepackage{color}
\usepackage{fancyvrb}
\newcommand{\VerbBar}{|}
\newcommand{\VERB}{\Verb[commandchars=\\\{\}]}
\DefineVerbatimEnvironment{Highlighting}{Verbatim}{commandchars=\\\{\}}
% Add ',fontsize=\small' for more characters per line
\usepackage{framed}
\definecolor{shadecolor}{RGB}{248,248,248}
\newenvironment{Shaded}{\begin{snugshade}}{\end{snugshade}}
\newcommand{\AlertTok}[1]{\textcolor[rgb]{0.94,0.16,0.16}{#1}}
\newcommand{\AnnotationTok}[1]{\textcolor[rgb]{0.56,0.35,0.01}{\textbf{\textit{#1}}}}
\newcommand{\AttributeTok}[1]{\textcolor[rgb]{0.13,0.29,0.53}{#1}}
\newcommand{\BaseNTok}[1]{\textcolor[rgb]{0.00,0.00,0.81}{#1}}
\newcommand{\BuiltInTok}[1]{#1}
\newcommand{\CharTok}[1]{\textcolor[rgb]{0.31,0.60,0.02}{#1}}
\newcommand{\CommentTok}[1]{\textcolor[rgb]{0.56,0.35,0.01}{\textit{#1}}}
\newcommand{\CommentVarTok}[1]{\textcolor[rgb]{0.56,0.35,0.01}{\textbf{\textit{#1}}}}
\newcommand{\ConstantTok}[1]{\textcolor[rgb]{0.56,0.35,0.01}{#1}}
\newcommand{\ControlFlowTok}[1]{\textcolor[rgb]{0.13,0.29,0.53}{\textbf{#1}}}
\newcommand{\DataTypeTok}[1]{\textcolor[rgb]{0.13,0.29,0.53}{#1}}
\newcommand{\DecValTok}[1]{\textcolor[rgb]{0.00,0.00,0.81}{#1}}
\newcommand{\DocumentationTok}[1]{\textcolor[rgb]{0.56,0.35,0.01}{\textbf{\textit{#1}}}}
\newcommand{\ErrorTok}[1]{\textcolor[rgb]{0.64,0.00,0.00}{\textbf{#1}}}
\newcommand{\ExtensionTok}[1]{#1}
\newcommand{\FloatTok}[1]{\textcolor[rgb]{0.00,0.00,0.81}{#1}}
\newcommand{\FunctionTok}[1]{\textcolor[rgb]{0.13,0.29,0.53}{\textbf{#1}}}
\newcommand{\ImportTok}[1]{#1}
\newcommand{\InformationTok}[1]{\textcolor[rgb]{0.56,0.35,0.01}{\textbf{\textit{#1}}}}
\newcommand{\KeywordTok}[1]{\textcolor[rgb]{0.13,0.29,0.53}{\textbf{#1}}}
\newcommand{\NormalTok}[1]{#1}
\newcommand{\OperatorTok}[1]{\textcolor[rgb]{0.81,0.36,0.00}{\textbf{#1}}}
\newcommand{\OtherTok}[1]{\textcolor[rgb]{0.56,0.35,0.01}{#1}}
\newcommand{\PreprocessorTok}[1]{\textcolor[rgb]{0.56,0.35,0.01}{\textit{#1}}}
\newcommand{\RegionMarkerTok}[1]{#1}
\newcommand{\SpecialCharTok}[1]{\textcolor[rgb]{0.81,0.36,0.00}{\textbf{#1}}}
\newcommand{\SpecialStringTok}[1]{\textcolor[rgb]{0.31,0.60,0.02}{#1}}
\newcommand{\StringTok}[1]{\textcolor[rgb]{0.31,0.60,0.02}{#1}}
\newcommand{\VariableTok}[1]{\textcolor[rgb]{0.00,0.00,0.00}{#1}}
\newcommand{\VerbatimStringTok}[1]{\textcolor[rgb]{0.31,0.60,0.02}{#1}}
\newcommand{\WarningTok}[1]{\textcolor[rgb]{0.56,0.35,0.01}{\textbf{\textit{#1}}}}
\usepackage{graphicx}
\makeatletter
\def\maxwidth{\ifdim\Gin@nat@width>\linewidth\linewidth\else\Gin@nat@width\fi}
\def\maxheight{\ifdim\Gin@nat@height>\textheight\textheight\else\Gin@nat@height\fi}
\makeatother
% Scale images if necessary, so that they will not overflow the page
% margins by default, and it is still possible to overwrite the defaults
% using explicit options in \includegraphics[width, height, ...]{}
\setkeys{Gin}{width=\maxwidth,height=\maxheight,keepaspectratio}
% Set default figure placement to htbp
\makeatletter
\def\fps@figure{htbp}
\makeatother
\setlength{\emergencystretch}{3em} % prevent overfull lines
\providecommand{\tightlist}{%
  \setlength{\itemsep}{0pt}\setlength{\parskip}{0pt}}
\setcounter{secnumdepth}{-\maxdimen} % remove section numbering
\usepackage{booktabs}
\usepackage{longtable}
\usepackage{array}
\usepackage{multirow}
\usepackage{wrapfig}
\usepackage{float}
\usepackage{colortbl}
\usepackage{pdflscape}
\usepackage{tabu}
\usepackage{threeparttable}
\usepackage{threeparttablex}
\usepackage[normalem]{ulem}
\usepackage{makecell}
\usepackage{xcolor}
\ifLuaTeX
  \usepackage{selnolig}  % disable illegal ligatures
\fi
\IfFileExists{bookmark.sty}{\usepackage{bookmark}}{\usepackage{hyperref}}
\IfFileExists{xurl.sty}{\usepackage{xurl}}{} % add URL line breaks if available
\urlstyle{same}
\hypersetup{
  pdftitle={Data Preprocessing},
  pdfauthor={J. Lecumberri, B. Fité \& OBustos},
  hidelinks,
  pdfcreator={LaTeX via pandoc}}

\title{Data Preprocessing}
\author{J. Lecumberri, B. Fité \& OBustos}
\date{2024-04-09}

\begin{document}
\maketitle

{
\setcounter{tocdepth}{5}
\tableofcontents
}
\hypertarget{libraries-loading}{%
\section{Libraries loading}\label{libraries-loading}}

\begin{Shaded}
\begin{Highlighting}[]
\FunctionTok{library}\NormalTok{(ggplot2)}
\FunctionTok{library}\NormalTok{(stringr)}
\FunctionTok{library}\NormalTok{(readxl)}
\FunctionTok{library}\NormalTok{(data.table)}
\FunctionTok{library}\NormalTok{(openxlsx)}
\FunctionTok{library}\NormalTok{(knitr)}
\FunctionTok{library}\NormalTok{(kableExtra)}
\FunctionTok{library}\NormalTok{(RColorBrewer)}
\FunctionTok{library}\NormalTok{(scales)}
\FunctionTok{library}\NormalTok{(dplyr)}
\end{Highlighting}
\end{Shaded}

\begin{verbatim}
## 
## Attaching package: 'dplyr'
\end{verbatim}

\begin{verbatim}
## The following object is masked from 'package:kableExtra':
## 
##     group_rows
\end{verbatim}

\begin{verbatim}
## The following objects are masked from 'package:data.table':
## 
##     between, first, last
\end{verbatim}

\begin{verbatim}
## The following objects are masked from 'package:stats':
## 
##     filter, lag
\end{verbatim}

\begin{verbatim}
## The following objects are masked from 'package:base':
## 
##     intersect, setdiff, setequal, union
\end{verbatim}

\begin{Shaded}
\begin{Highlighting}[]
\FunctionTok{library}\NormalTok{(gridExtra)}
\end{Highlighting}
\end{Shaded}

\begin{verbatim}
## 
## Attaching package: 'gridExtra'
\end{verbatim}

\begin{verbatim}
## The following object is masked from 'package:dplyr':
## 
##     combine
\end{verbatim}

\begin{Shaded}
\begin{Highlighting}[]
\FunctionTok{library}\NormalTok{(plotly)}
\end{Highlighting}
\end{Shaded}

\begin{verbatim}
## 
## Attaching package: 'plotly'
\end{verbatim}

\begin{verbatim}
## The following object is masked from 'package:ggplot2':
## 
##     last_plot
\end{verbatim}

\begin{verbatim}
## The following object is masked from 'package:stats':
## 
##     filter
\end{verbatim}

\begin{verbatim}
## The following object is masked from 'package:graphics':
## 
##     layout
\end{verbatim}

\begin{Shaded}
\begin{Highlighting}[]
\FunctionTok{library}\NormalTok{(car)}
\end{Highlighting}
\end{Shaded}

\begin{verbatim}
## Loading required package: carData
\end{verbatim}

\begin{verbatim}
## 
## Attaching package: 'car'
\end{verbatim}

\begin{verbatim}
## The following object is masked from 'package:dplyr':
## 
##     recode
\end{verbatim}

\hypertarget{data-retrieval}{%
\section{Data retrieval}\label{data-retrieval}}

Data of the project will be stored in variable \texttt{data}.

\begin{Shaded}
\begin{Highlighting}[]
\NormalTok{cd }\OtherTok{=} \FunctionTok{getwd}\NormalTok{()}
\NormalTok{parent\_dir }\OtherTok{\textless{}{-}} \FunctionTok{dirname}\NormalTok{(cd)}
\NormalTok{file\_path\_data }\OtherTok{=} \FunctionTok{file.path}\NormalTok{(parent\_dir, }\StringTok{"data"}\NormalTok{)}

\NormalTok{data }\OtherTok{=} \FunctionTok{read.table}\NormalTok{(}\FunctionTok{file.path}\NormalTok{(file\_path\_data, }\StringTok{"heart\_data.csv"}\NormalTok{), }\AttributeTok{header =}\NormalTok{ T, }\AttributeTok{sep =} \StringTok{","}\NormalTok{)}
\end{Highlighting}
\end{Shaded}

\hypertarget{initial-data-exploration}{%
\subsection{Initial data exploration}\label{initial-data-exploration}}

Our \texttt{data} is initially comprised of 70000 observations and 14
variables

\begin{Shaded}
\begin{Highlighting}[]
\FunctionTok{dim}\NormalTok{(data)}
\end{Highlighting}
\end{Shaded}

\begin{verbatim}
## [1] 70000    14
\end{verbatim}

\begin{Shaded}
\begin{Highlighting}[]
\FunctionTok{cat}\NormalTok{(}\StringTok{"Variable names: }\SpecialCharTok{\textbackslash{}n}\StringTok{"}\NormalTok{, }\FunctionTok{colnames}\NormalTok{(data))}
\end{Highlighting}
\end{Shaded}

\begin{verbatim}
## Variable names: 
##  index id age gender height weight ap_hi ap_lo cholesterol gluc smoke alco active cardio
\end{verbatim}

These variables are:

\begin{itemize}
\tightlist
\item
  \textbf{index} and \textbf{id}: subject information. Only id will be
  kept.
\item
  \textbf{age}: age in days of each subject.
\item
  \textbf{gender}: binary variable (0 and 1) with gender information.
\item
  \textbf{height}: data in cm.
\item
  \textbf{weight}: data in kg.
\item
  \textbf{ap\_hi}: systolic blood pressure reading.
\item
  \textbf{lo\_hi}: diastolic blood pressure reading.
\item
  \textbf{cholesterol}: total cholesterol level on scale 0-5. Each unit
  denotes decrease by 20 mg/dL.
\item
  \textbf{gluc}: glucose level read as mmol/l.
\item
  \textbf{smoke}: wether subject smokes (1) or not (0).
\item
  \textbf{alco}: whether subject drinks alcohol (1) or not (0).
\item
  \textbf{active}: whether subject is active (1) or not (0).
\item
  \textbf{cardio}: whether subject suffers from cardiovascular disease
  (1) or not (0).
\end{itemize}

\hypertarget{data-processing}{%
\section{Data processing}\label{data-processing}}

Before any analysis it is important to perform the necessary
modifications, additions and filtering steps in order to correct units,
missing values, add informative data or remove outliers.

\hypertarget{initial-variable-modifications-and-additions}{%
\subsection{Initial variable modifications and
additions}\label{initial-variable-modifications-and-additions}}

First of all we will correct \texttt{age} variable and put it in years
units:

\begin{Shaded}
\begin{Highlighting}[]
\NormalTok{data}\SpecialCharTok{$}\NormalTok{age }\OtherTok{=} \FunctionTok{round}\NormalTok{(data}\SpecialCharTok{$}\NormalTok{age }\SpecialCharTok{/} \DecValTok{365}\NormalTok{, }\DecValTok{2}\NormalTok{) }
\end{Highlighting}
\end{Shaded}

Addition of BMI variable, computed as follows:

\[BMI = \frac{weight_{kg}}{height_{m}^2}\]

\begin{Shaded}
\begin{Highlighting}[]
\NormalTok{data}\SpecialCharTok{$}\NormalTok{BMI }\OtherTok{=} \FunctionTok{round}\NormalTok{(data}\SpecialCharTok{$}\NormalTok{weight }\SpecialCharTok{/}\NormalTok{ (data}\SpecialCharTok{$}\NormalTok{height }\SpecialCharTok{/} \DecValTok{100}\NormalTok{)}\SpecialCharTok{**}\DecValTok{2}\NormalTok{, }\DecValTok{2}\NormalTok{) }
\end{Highlighting}
\end{Shaded}

Removal of \texttt{id} and \texttt{index} variables as row id already
covers for this information:

\begin{Shaded}
\begin{Highlighting}[]
\NormalTok{data }\OtherTok{=}\NormalTok{ data[,}\FunctionTok{c}\NormalTok{(}\SpecialCharTok{{-}}\DecValTok{1}\NormalTok{,}\SpecialCharTok{{-}}\DecValTok{2}\NormalTok{)] }
\end{Highlighting}
\end{Shaded}

\hypertarget{filtering-removal-of-outliers}{%
\subsection{Filtering: removal of
outliers}\label{filtering-removal-of-outliers}}

By inspecting how the different variables are:

\hypertarget{height---weight}{%
\subsubsection{Height - Weight}\label{height---weight}}

\begin{Shaded}
\begin{Highlighting}[]
\FunctionTok{boxplot}\NormalTok{(data}\SpecialCharTok{$}\NormalTok{height, }\AttributeTok{main =} \StringTok{"HEIGHT"}\NormalTok{)}
\end{Highlighting}
\end{Shaded}

\includegraphics{DataPreprocessingReport_files/figure-latex/unnamed-chunk-7-1.pdf}

\begin{Shaded}
\begin{Highlighting}[]
\FunctionTok{boxplot}\NormalTok{(data}\SpecialCharTok{$}\NormalTok{weight, }\AttributeTok{main =} \StringTok{"WEIGHT"}\NormalTok{)}
\end{Highlighting}
\end{Shaded}

\includegraphics{DataPreprocessingReport_files/figure-latex/unnamed-chunk-7-2.pdf}

\begin{Shaded}
\begin{Highlighting}[]
\FunctionTok{ggplot}\NormalTok{(data, }\FunctionTok{aes}\NormalTok{(}\AttributeTok{x =}\NormalTok{ height, }\AttributeTok{y =}\NormalTok{ weight)) }\SpecialCharTok{+}
  \FunctionTok{geom\_point}\NormalTok{() }\SpecialCharTok{+}
  \FunctionTok{labs}\NormalTok{(}\AttributeTok{x =} \StringTok{"height"}\NormalTok{, }\AttributeTok{y =} \StringTok{"weight"}\NormalTok{) }\SpecialCharTok{+}
  \FunctionTok{theme\_minimal}\NormalTok{()}
\end{Highlighting}
\end{Shaded}

\includegraphics{DataPreprocessingReport_files/figure-latex/unnamed-chunk-7-3.pdf}

We can see how in \texttt{height} and \texttt{weight} have a lot of
clears outliers and clear data errors:

\begin{Shaded}
\begin{Highlighting}[]
\FunctionTok{cat}\NormalTok{(}\StringTok{"Maximum value of height"}\NormalTok{, }\FunctionTok{max}\NormalTok{(data}\SpecialCharTok{$}\NormalTok{height),}\StringTok{"}\SpecialCharTok{\textbackslash{}n}\StringTok{"}\NormalTok{)}
\end{Highlighting}
\end{Shaded}

\begin{verbatim}
## Maximum value of height 250
\end{verbatim}

\begin{Shaded}
\begin{Highlighting}[]
\FunctionTok{cat}\NormalTok{(}\StringTok{"Maximum value of weight"}\NormalTok{, }\FunctionTok{max}\NormalTok{(data}\SpecialCharTok{$}\NormalTok{weight),}\StringTok{"}\SpecialCharTok{\textbackslash{}n}\StringTok{"}\NormalTok{)}
\end{Highlighting}
\end{Shaded}

\begin{verbatim}
## Maximum value of weight 200
\end{verbatim}

\begin{Shaded}
\begin{Highlighting}[]
\FunctionTok{cat}\NormalTok{(}\StringTok{"Minumum value of height"}\NormalTok{, }\FunctionTok{min}\NormalTok{(data}\SpecialCharTok{$}\NormalTok{height),}\StringTok{"}\SpecialCharTok{\textbackslash{}n}\StringTok{"}\NormalTok{)}
\end{Highlighting}
\end{Shaded}

\begin{verbatim}
## Minumum value of height 55
\end{verbatim}

\begin{Shaded}
\begin{Highlighting}[]
\FunctionTok{cat}\NormalTok{(}\StringTok{"Minumum value of weight"}\NormalTok{, }\FunctionTok{min}\NormalTok{(data}\SpecialCharTok{$}\NormalTok{weight),}\StringTok{"}\SpecialCharTok{\textbackslash{}n}\StringTok{"}\NormalTok{)}
\end{Highlighting}
\end{Shaded}

\begin{verbatim}
## Minumum value of weight 10
\end{verbatim}

Filtering applied:

\begin{Shaded}
\begin{Highlighting}[]
\NormalTok{data }\OtherTok{\textless{}{-}}\NormalTok{ data[data}\SpecialCharTok{$}\NormalTok{height }\SpecialCharTok{\textgreater{}=} \DecValTok{130} \SpecialCharTok{\&}\NormalTok{ data}\SpecialCharTok{$}\NormalTok{height }\SpecialCharTok{\textless{}=} \DecValTok{230}\NormalTok{, ] }
\NormalTok{data }\OtherTok{\textless{}{-}}\NormalTok{ data[data}\SpecialCharTok{$}\NormalTok{weight }\SpecialCharTok{\textgreater{}=} \DecValTok{45}\NormalTok{,]}
\end{Highlighting}
\end{Shaded}

Once filtered is looks as follows:

\begin{Shaded}
\begin{Highlighting}[]
\FunctionTok{ggplot}\NormalTok{(data, }\FunctionTok{aes}\NormalTok{(}\AttributeTok{x =}\NormalTok{ height, }\AttributeTok{y =}\NormalTok{ weight)) }\SpecialCharTok{+}
  \FunctionTok{geom\_point}\NormalTok{() }\SpecialCharTok{+}
  \FunctionTok{labs}\NormalTok{(}\AttributeTok{x =} \StringTok{"height"}\NormalTok{, }\AttributeTok{y =} \StringTok{"weight"}\NormalTok{) }\SpecialCharTok{+}
  \FunctionTok{theme\_minimal}\NormalTok{()}
\end{Highlighting}
\end{Shaded}

\includegraphics{DataPreprocessingReport_files/figure-latex/unnamed-chunk-10-1.pdf}

\hypertarget{systolic-and-diastolic-pressures}{%
\subsubsection{Systolic and diastolic
pressures}\label{systolic-and-diastolic-pressures}}

First of all let's check how these two variables are distributed:

\begin{Shaded}
\begin{Highlighting}[]
\FunctionTok{ggplot}\NormalTok{(data, }\FunctionTok{aes}\NormalTok{(}\AttributeTok{x =}\NormalTok{ ap\_hi, }\AttributeTok{y =}\NormalTok{ ap\_lo)) }\SpecialCharTok{+}
  \FunctionTok{geom\_point}\NormalTok{() }\SpecialCharTok{+}
  \FunctionTok{labs}\NormalTok{(}\AttributeTok{x =} \StringTok{"ap\_hi"}\NormalTok{, }\AttributeTok{y =} \StringTok{"ap\_lo"}\NormalTok{) }\SpecialCharTok{+}
  \FunctionTok{theme\_minimal}\NormalTok{()}
\end{Highlighting}
\end{Shaded}

\includegraphics{DataPreprocessingReport_files/figure-latex/unnamed-chunk-11-1.pdf}

We can observe clear outliers:

\begin{Shaded}
\begin{Highlighting}[]
\FunctionTok{cat}\NormalTok{(}\StringTok{"Maximum value of systolic P"}\NormalTok{, }\FunctionTok{max}\NormalTok{(data}\SpecialCharTok{$}\NormalTok{ap\_hi),}\StringTok{"}\SpecialCharTok{\textbackslash{}n}\StringTok{"}\NormalTok{)}
\end{Highlighting}
\end{Shaded}

\begin{verbatim}
## Maximum value of systolic P 16020
\end{verbatim}

\begin{Shaded}
\begin{Highlighting}[]
\FunctionTok{cat}\NormalTok{(}\StringTok{"Maximum value of diastolic P"}\NormalTok{, }\FunctionTok{max}\NormalTok{(data}\SpecialCharTok{$}\NormalTok{ap\_lo),}\StringTok{"}\SpecialCharTok{\textbackslash{}n}\StringTok{"}\NormalTok{)}
\end{Highlighting}
\end{Shaded}

\begin{verbatim}
## Maximum value of diastolic P 11000
\end{verbatim}

\begin{Shaded}
\begin{Highlighting}[]
\FunctionTok{cat}\NormalTok{(}\StringTok{"Minumum value of systolic P"}\NormalTok{, }\FunctionTok{min}\NormalTok{(data}\SpecialCharTok{$}\NormalTok{ap\_hi),}\StringTok{"}\SpecialCharTok{\textbackslash{}n}\StringTok{"}\NormalTok{)}
\end{Highlighting}
\end{Shaded}

\begin{verbatim}
## Minumum value of systolic P -150
\end{verbatim}

\begin{Shaded}
\begin{Highlighting}[]
\FunctionTok{cat}\NormalTok{(}\StringTok{"Minumum value of diastolic P"}\NormalTok{, }\FunctionTok{min}\NormalTok{(data}\SpecialCharTok{$}\NormalTok{ap\_lo),}\StringTok{"}\SpecialCharTok{\textbackslash{}n}\StringTok{"}\NormalTok{)}
\end{Highlighting}
\end{Shaded}

\begin{verbatim}
## Minumum value of diastolic P -70
\end{verbatim}

So we will filter data by sources found for normal values of systolic
and diastolic pressure

\begin{Shaded}
\begin{Highlighting}[]
\NormalTok{data }\OtherTok{\textless{}{-}}\NormalTok{ data[data}\SpecialCharTok{$}\NormalTok{ap\_hi }\SpecialCharTok{\textless{}=} \DecValTok{200} \SpecialCharTok{\&}\NormalTok{ data}\SpecialCharTok{$}\NormalTok{ap\_lo }\SpecialCharTok{\textless{}=} \DecValTok{140}\NormalTok{, ]}
\NormalTok{data }\OtherTok{\textless{}{-}}\NormalTok{ data[data}\SpecialCharTok{$}\NormalTok{ap\_hi }\SpecialCharTok{\textgreater{}=} \DecValTok{80} \SpecialCharTok{\&}\NormalTok{ data}\SpecialCharTok{$}\NormalTok{ap\_lo }\SpecialCharTok{\textgreater{}=} \DecValTok{50}\NormalTok{, ]}

\FunctionTok{ggplot}\NormalTok{(data, }\FunctionTok{aes}\NormalTok{(}\AttributeTok{x =}\NormalTok{ ap\_hi, }\AttributeTok{y =}\NormalTok{ ap\_lo)) }\SpecialCharTok{+}
  \FunctionTok{geom\_point}\NormalTok{() }\SpecialCharTok{+}
  \FunctionTok{labs}\NormalTok{(}\AttributeTok{x =} \StringTok{"ap\_hi"}\NormalTok{, }\AttributeTok{y =} \StringTok{"ap\_lo"}\NormalTok{) }\SpecialCharTok{+}
  \FunctionTok{theme\_minimal}\NormalTok{()}
\end{Highlighting}
\end{Shaded}

\includegraphics{DataPreprocessingReport_files/figure-latex/unnamed-chunk-13-1.pdf}

\hypertarget{gender-detection}{%
\subsubsection{Gender detection}\label{gender-detection}}

We have a binary result for gender: 1 and 2, but we need to characterize
it correctly by male and female.

\begin{Shaded}
\begin{Highlighting}[]
\NormalTok{weights\_1 }\OtherTok{=}\NormalTok{ data[data}\SpecialCharTok{$}\NormalTok{gender }\SpecialCharTok{==} \DecValTok{1}\NormalTok{,]}\SpecialCharTok{$}\NormalTok{weight}
\NormalTok{weights\_2 }\OtherTok{=}\NormalTok{ data[data}\SpecialCharTok{$}\NormalTok{gender }\SpecialCharTok{==} \DecValTok{2}\NormalTok{,]}\SpecialCharTok{$}\NormalTok{weight}

\FunctionTok{cat}\NormalTok{(}\StringTok{"Mean weight value for gender = 1"}\NormalTok{,}\FunctionTok{mean}\NormalTok{(weights\_1),}\StringTok{"}\SpecialCharTok{\textbackslash{}n}\StringTok{"}\NormalTok{)}
\end{Highlighting}
\end{Shaded}

\begin{verbatim}
## Mean weight value for gender = 1 72.65673
\end{verbatim}

\begin{Shaded}
\begin{Highlighting}[]
\FunctionTok{cat}\NormalTok{(}\StringTok{"Mean weight value for gender = 2"}\NormalTok{,}\FunctionTok{mean}\NormalTok{(weights\_2),}\StringTok{"}\SpecialCharTok{\textbackslash{}n\textbackslash{}n}\StringTok{"}\NormalTok{)}
\end{Highlighting}
\end{Shaded}

\begin{verbatim}
## Mean weight value for gender = 2 77.19945
\end{verbatim}

\begin{Shaded}
\begin{Highlighting}[]
\NormalTok{heights\_1 }\OtherTok{=}\NormalTok{ data[data}\SpecialCharTok{$}\NormalTok{gender }\SpecialCharTok{==} \DecValTok{1}\NormalTok{,]}\SpecialCharTok{$}\NormalTok{height}
\NormalTok{heights\_2 }\OtherTok{=}\NormalTok{ data[data}\SpecialCharTok{$}\NormalTok{gender }\SpecialCharTok{==} \DecValTok{2}\NormalTok{,]}\SpecialCharTok{$}\NormalTok{height}

\FunctionTok{cat}\NormalTok{(}\StringTok{"Mean height value for gender = 1"}\NormalTok{,}\FunctionTok{mean}\NormalTok{(heights\_1),}\StringTok{"}\SpecialCharTok{\textbackslash{}n}\StringTok{"}\NormalTok{)}
\end{Highlighting}
\end{Shaded}

\begin{verbatim}
## Mean height value for gender = 1 161.4979
\end{verbatim}

\begin{Shaded}
\begin{Highlighting}[]
\FunctionTok{cat}\NormalTok{(}\StringTok{"Mean height value for gender = 2"}\NormalTok{,}\FunctionTok{mean}\NormalTok{(heights\_2))}
\end{Highlighting}
\end{Shaded}

\begin{verbatim}
## Mean height value for gender = 2 170.0443
\end{verbatim}

So knowing that males, even more in a dataset of this size, have in
avergare higher height and weight:

\begin{Shaded}
\begin{Highlighting}[]
\NormalTok{data}\SpecialCharTok{$}\NormalTok{gender }\OtherTok{\textless{}{-}} \FunctionTok{ifelse}\NormalTok{(data}\SpecialCharTok{$}\NormalTok{gender }\SpecialCharTok{==} \DecValTok{1}\NormalTok{, }\StringTok{"F"}\NormalTok{, }\StringTok{"M"}\NormalTok{)}
\NormalTok{data}\SpecialCharTok{$}\NormalTok{gender }\OtherTok{\textless{}{-}} \FunctionTok{as.factor}\NormalTok{(data}\SpecialCharTok{$}\NormalTok{gender)}
\end{Highlighting}
\end{Shaded}

\begin{Shaded}
\begin{Highlighting}[]
\FunctionTok{ggplot}\NormalTok{(data, }\FunctionTok{aes}\NormalTok{(}\AttributeTok{x =}\NormalTok{ weight, }\AttributeTok{y =}\NormalTok{ height, }\AttributeTok{color =}\NormalTok{ gender)) }\SpecialCharTok{+}
  \FunctionTok{geom\_point}\NormalTok{() }\SpecialCharTok{+}
  \FunctionTok{labs}\NormalTok{(}\AttributeTok{x =} \StringTok{"Weight"}\NormalTok{, }\AttributeTok{y =} \StringTok{"Height"}\NormalTok{, }\AttributeTok{color =} \StringTok{"Gender"}\NormalTok{) }\SpecialCharTok{+}
  \FunctionTok{theme\_minimal}\NormalTok{()}
\end{Highlighting}
\end{Shaded}

\includegraphics{DataPreprocessingReport_files/figure-latex/unnamed-chunk-16-1.pdf}

\begin{Shaded}
\begin{Highlighting}[]
\FunctionTok{par}\NormalTok{(}\AttributeTok{mfrow=}\FunctionTok{c}\NormalTok{(}\DecValTok{1}\NormalTok{,}\DecValTok{3}\NormalTok{))}
\FunctionTok{plot}\NormalTok{(}\FunctionTok{density}\NormalTok{(data}\SpecialCharTok{$}\NormalTok{age), }\AttributeTok{main =} \StringTok{"Age"}\NormalTok{)}
\FunctionTok{plot}\NormalTok{(}\FunctionTok{density}\NormalTok{(data}\SpecialCharTok{$}\NormalTok{height), }\AttributeTok{main =} \StringTok{"Height"}\NormalTok{)}
\FunctionTok{plot}\NormalTok{(}\FunctionTok{density}\NormalTok{(data}\SpecialCharTok{$}\NormalTok{weight), }\AttributeTok{main =} \StringTok{"Weight"}\NormalTok{)}
\end{Highlighting}
\end{Shaded}

\includegraphics{DataPreprocessingReport_files/figure-latex/unnamed-chunk-17-1.pdf}

\begin{Shaded}
\begin{Highlighting}[]
\FunctionTok{plot}\NormalTok{(}\FunctionTok{density}\NormalTok{(data}\SpecialCharTok{$}\NormalTok{ap\_hi), }\AttributeTok{main =} \StringTok{"Ap Hi"}\NormalTok{)}
\FunctionTok{plot}\NormalTok{(}\FunctionTok{density}\NormalTok{(data}\SpecialCharTok{$}\NormalTok{ap\_lo), }\AttributeTok{main =} \StringTok{"Ap Lo"}\NormalTok{)}
\FunctionTok{plot}\NormalTok{(}\FunctionTok{density}\NormalTok{(data}\SpecialCharTok{$}\NormalTok{BMI), }\AttributeTok{main =} \StringTok{"BMI"}\NormalTok{)}
\end{Highlighting}
\end{Shaded}

\includegraphics{DataPreprocessingReport_files/figure-latex/unnamed-chunk-17-2.pdf}

We have normal apparent normal distribution across all these continuous
variables.

\hypertarget{factorization-of-discrete-variables}{%
\subsubsection{Factorization of discrete
variables}\label{factorization-of-discrete-variables}}

Now for the finale variables, those that are discrete over certain
ranges of values, specified in the variables descriptions, we can study
their representations and factor them.

\begin{Shaded}
\begin{Highlighting}[]
\FunctionTok{table}\NormalTok{(data}\SpecialCharTok{$}\NormalTok{cholesterol)}
\end{Highlighting}
\end{Shaded}

\begin{verbatim}
## 
##     1     2     3 
## 51200  9239  7840
\end{verbatim}

\begin{Shaded}
\begin{Highlighting}[]
\FunctionTok{table}\NormalTok{(data}\SpecialCharTok{$}\NormalTok{gluc)}
\end{Highlighting}
\end{Shaded}

\begin{verbatim}
## 
##     1     2     3 
## 58042  5036  5201
\end{verbatim}

\begin{Shaded}
\begin{Highlighting}[]
\FunctionTok{table}\NormalTok{(data}\SpecialCharTok{$}\NormalTok{smoke)}
\end{Highlighting}
\end{Shaded}

\begin{verbatim}
## 
##     0     1 
## 62254  6025
\end{verbatim}

\begin{Shaded}
\begin{Highlighting}[]
\FunctionTok{table}\NormalTok{(data}\SpecialCharTok{$}\NormalTok{alco)}
\end{Highlighting}
\end{Shaded}

\begin{verbatim}
## 
##     0     1 
## 64618  3661
\end{verbatim}

\begin{Shaded}
\begin{Highlighting}[]
\FunctionTok{table}\NormalTok{(data}\SpecialCharTok{$}\NormalTok{active)}
\end{Highlighting}
\end{Shaded}

\begin{verbatim}
## 
##     0     1 
## 13432 54847
\end{verbatim}

\begin{Shaded}
\begin{Highlighting}[]
\FunctionTok{table}\NormalTok{(data}\SpecialCharTok{$}\NormalTok{cardio)}
\end{Highlighting}
\end{Shaded}

\begin{verbatim}
## 
##     0     1 
## 34439 33840
\end{verbatim}

\begin{Shaded}
\begin{Highlighting}[]
\NormalTok{data}\SpecialCharTok{$}\NormalTok{cholesterol }\OtherTok{\textless{}{-}} \FunctionTok{as.factor}\NormalTok{(data}\SpecialCharTok{$}\NormalTok{cholesterol)}
\NormalTok{data}\SpecialCharTok{$}\NormalTok{gluc }\OtherTok{\textless{}{-}} \FunctionTok{as.factor}\NormalTok{(data}\SpecialCharTok{$}\NormalTok{gluc)}
\NormalTok{data}\SpecialCharTok{$}\NormalTok{smoke }\OtherTok{\textless{}{-}} \FunctionTok{as.factor}\NormalTok{(data}\SpecialCharTok{$}\NormalTok{smoke)}
\NormalTok{data}\SpecialCharTok{$}\NormalTok{alco }\OtherTok{\textless{}{-}} \FunctionTok{as.factor}\NormalTok{(data}\SpecialCharTok{$}\NormalTok{alco)}
\NormalTok{data}\SpecialCharTok{$}\NormalTok{cardio }\OtherTok{\textless{}{-}} \FunctionTok{as.factor}\NormalTok{(data}\SpecialCharTok{$}\NormalTok{cardio)}
\end{Highlighting}
\end{Shaded}

\begin{Shaded}
\begin{Highlighting}[]
\NormalTok{p1 }\OtherTok{\textless{}{-}} \FunctionTok{ggplot}\NormalTok{(data, }\FunctionTok{aes}\NormalTok{(}\AttributeTok{x =}\NormalTok{ cholesterol)) }\SpecialCharTok{+}
  \FunctionTok{geom\_bar}\NormalTok{() }\SpecialCharTok{+}
  \FunctionTok{labs}\NormalTok{(}\AttributeTok{title =} \StringTok{"Cholesterol Levels"}\NormalTok{,}
       \AttributeTok{x =} \StringTok{"Cholesterol Levels"}\NormalTok{,}
       \AttributeTok{y =} \StringTok{"Count"}\NormalTok{) }\SpecialCharTok{+}
  \FunctionTok{theme}\NormalTok{(}\AttributeTok{plot.title =} \FunctionTok{element\_text}\NormalTok{(}\AttributeTok{size =} \DecValTok{10}\NormalTok{)) }\CommentTok{\# Adjust size here}

\NormalTok{p2 }\OtherTok{\textless{}{-}} \FunctionTok{ggplot}\NormalTok{(data, }\FunctionTok{aes}\NormalTok{(}\AttributeTok{x =}\NormalTok{ gluc)) }\SpecialCharTok{+}
  \FunctionTok{geom\_bar}\NormalTok{() }\SpecialCharTok{+}
  \FunctionTok{labs}\NormalTok{(}\AttributeTok{title =} \StringTok{"Glucose Levels"}\NormalTok{,}
       \AttributeTok{x =} \StringTok{"Glucose Levels"}\NormalTok{,}
       \AttributeTok{y =} \StringTok{"Count"}\NormalTok{) }\SpecialCharTok{+}
  \FunctionTok{theme}\NormalTok{(}\AttributeTok{plot.title =} \FunctionTok{element\_text}\NormalTok{(}\AttributeTok{size =} \DecValTok{10}\NormalTok{)) }\CommentTok{\# Adjust size here}

\NormalTok{p3 }\OtherTok{\textless{}{-}} \FunctionTok{ggplot}\NormalTok{(data, }\FunctionTok{aes}\NormalTok{(}\AttributeTok{x =}\NormalTok{ smoke)) }\SpecialCharTok{+}
  \FunctionTok{geom\_bar}\NormalTok{() }\SpecialCharTok{+}
  \FunctionTok{labs}\NormalTok{(}\AttributeTok{title =} \StringTok{"Smoke Status"}\NormalTok{,}
       \AttributeTok{x =} \StringTok{"Smoke Levels"}\NormalTok{,}
       \AttributeTok{y =} \StringTok{"Count"}\NormalTok{) }\SpecialCharTok{+}
  \FunctionTok{theme}\NormalTok{(}\AttributeTok{plot.title =} \FunctionTok{element\_text}\NormalTok{(}\AttributeTok{size =} \DecValTok{10}\NormalTok{)) }\CommentTok{\# Adjust size here}

\NormalTok{p4 }\OtherTok{\textless{}{-}} \FunctionTok{ggplot}\NormalTok{(data, }\FunctionTok{aes}\NormalTok{(}\AttributeTok{x =}\NormalTok{ alco)) }\SpecialCharTok{+}
  \FunctionTok{geom\_bar}\NormalTok{() }\SpecialCharTok{+}
  \FunctionTok{labs}\NormalTok{(}\AttributeTok{title =} \StringTok{"Alcohol Status"}\NormalTok{,}
       \AttributeTok{x =} \StringTok{"Alcohol Levels"}\NormalTok{,}
       \AttributeTok{y =} \StringTok{"Count"}\NormalTok{) }\SpecialCharTok{+}
  \FunctionTok{theme}\NormalTok{(}\AttributeTok{plot.title =} \FunctionTok{element\_text}\NormalTok{(}\AttributeTok{size =} \DecValTok{10}\NormalTok{)) }\CommentTok{\# Adjust size here}

\NormalTok{p5 }\OtherTok{\textless{}{-}} \FunctionTok{ggplot}\NormalTok{(data, }\FunctionTok{aes}\NormalTok{(}\AttributeTok{x =}\NormalTok{ active)) }\SpecialCharTok{+}
  \FunctionTok{geom\_bar}\NormalTok{() }\SpecialCharTok{+}
  \FunctionTok{labs}\NormalTok{(}\AttributeTok{title =} \StringTok{"Active Status"}\NormalTok{,}
       \AttributeTok{x =} \StringTok{"Active Levels"}\NormalTok{,}
       \AttributeTok{y =} \StringTok{"Count"}\NormalTok{) }\SpecialCharTok{+}
  \FunctionTok{theme}\NormalTok{(}\AttributeTok{plot.title =} \FunctionTok{element\_text}\NormalTok{(}\AttributeTok{size =} \DecValTok{10}\NormalTok{)) }\CommentTok{\# Adjust size here}

\NormalTok{p6 }\OtherTok{\textless{}{-}} \FunctionTok{ggplot}\NormalTok{(data, }\FunctionTok{aes}\NormalTok{(}\AttributeTok{x =}\NormalTok{ cardio)) }\SpecialCharTok{+}
  \FunctionTok{geom\_bar}\NormalTok{() }\SpecialCharTok{+}
  \FunctionTok{labs}\NormalTok{(}\AttributeTok{title =} \StringTok{"Cardio Output Levels"}\NormalTok{,}
       \AttributeTok{x =} \StringTok{"Cardio Levels"}\NormalTok{,}
       \AttributeTok{y =} \StringTok{"Count"}\NormalTok{) }\SpecialCharTok{+}
  \FunctionTok{theme}\NormalTok{(}\AttributeTok{plot.title =} \FunctionTok{element\_text}\NormalTok{(}\AttributeTok{size =} \DecValTok{10}\NormalTok{)) }\CommentTok{\# Adjust size here}

\FunctionTok{grid.arrange}\NormalTok{(p1, p2, p3, p4, p5, p6, }\AttributeTok{nrow =} \DecValTok{2}\NormalTok{, }\AttributeTok{ncol =} \DecValTok{3}\NormalTok{)}
\end{Highlighting}
\end{Shaded}

\includegraphics{DataPreprocessingReport_files/figure-latex/unnamed-chunk-19-1.pdf}

\end{document}
